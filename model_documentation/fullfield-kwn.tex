\documentclass[11pt]{scrartcl}
\usepackage[usenames,dvipsnames,pdftex]{xcolor}
\usepackage{amsmath,amssymb,amsfonts}
\usepackage{bm}
\usepackage[load-configurations=version-1]{siunitx}
\usepackage[numbers,sort&compress]{natbib}
\usepackage{graphicx}


\usepackage{imakeidx}

\makeindex

\newcommand{\question}[1]{\textcolor{Red}{#1}}
\newcommand{\answer}[1]{\textcolor{Green}{#1}}
\newcommand{\note}[1]{\textcolor{CornflowerBlue}{#1}}

% REFERENCES
\usepackage[]{hyperref}
\newcommand{\eref}[1]{Eq.~\eqref{#1}}
\newcommand{\Eref}[1]{Equation~\eqref{#1}}
\newcommand{\fref}[1]{Fig.~\ref{#1}}
\newcommand{\Fref}[1]{Figure~\ref{#1}}
\newcommand{\tref}[1]{Tab.~\ref{#1}}
\newcommand{\Tref}[1]{Table~\ref{#1}}
\newcommand{\sref}[1]{section~\ref{#1}}
\newcommand{\Sref}[1]{Section~\ref{#1}}


% ABBREVIATIONS

\newcommand{\ie}{\textit{i.e.}}
\newcommand{\eg}{\textit{e.g.}}
\newcommand{\cf}{\textit{cf.}}


% NAMES 

\newcommand{\Euler}{\textsc{Euler}}
\newcommand{\Gauss}{\textsc{Gauss}}
\newcommand{\Kroener}{\textsc{Kr\"oner}}
\newcommand{\Nye}{\textsc{Nye}}
\newcommand{\Burgers}{\textsc{Burgers}}
\newcommand{\PK}{\textsc{Piola-Kirchhoff}}

% PHYSICAL CONSTANTS

\newcommand{\kB}{\ensuremath{k_\text{B}}}


% TENSORS 

\newcommand{\field}[1]{\ensuremath{\mathcal{#1}}}
\newcommand{\tnsr}[1]{\ensuremath{\mathbf{#1}}}
\newcommand{\vctr}[1]{\ensuremath{\bm{#1}}}


% SPECIAL TENSORS

\newcommand{\identity}{\ensuremath{\tnsr I}}
\newcommand{\levi}{\ensuremath{\epsilon}}


% FUNCTIONS 

\newcommand{\abs}[1]{\ensuremath{\left|{#1}\right|}}
\newcommand{\positive}[1]{\ensuremath{\left\lceil{#1}\right\rceil}}
\newcommand{\negative}[1]{\ensuremath{\left\lfloor{#1}\right\rfloor}}
\newcommand{\norm}[1]{\ensuremath{\left|\left|{#1}\right|\right|}}
\newcommand{\transpose}[1]{\ensuremath{{#1}^{\mathrm T}}}
\newcommand{\inverse}[1]{\ensuremath{{#1}^{-1}}}
\newcommand{\invtranspose}[1]{\ensuremath{{#1}^{\mathrm{-T}}}}
\newcommand{\inc}[1]{\ensuremath{\text d\,{#1}}}
\newcommand{\sign}[1]{\ensuremath{\operatorname{sign}{#1}}}
\newcommand{\signb}[1]{\ensuremath{\operatorname{sign}\left({#1}\right)}}
\newcommand{\grad}[2][]{\ensuremath{\operatorname{grad}_{#1}{#2}}}
\newcommand{\gradb}[2][]{\ensuremath{\operatorname{grad}_{#1}\left({#2}\right)}}
\newcommand{\divergence}[1][]{\ensuremath{\operatorname{div}{#1}}}
\newcommand{\divergenceb}[1]{\ensuremath{\operatorname{div}\left({#1}\right)}}
\newcommand{\curl}[1][]{\ensuremath{\operatorname{curl}{#1}}}
\newcommand{\curlb}[1]{\ensuremath{\operatorname{curl}\left({#1}\right)}}
\newcommand{\expb}[1]{\ensuremath{\exp{\left({#1}\right)}}}
\newcommand{\totalder}[2]{\ensuremath{\frac{\inc{#1}}{\inc{#2}}}}
\newcommand{\partialder}[2]{\ensuremath{\frac{\partial{#1}}{\partial{#2}}}}
\newcommand{\partialderb}[2]{\ensuremath{\partial_{#2}\left({#1}\right)}}
\newcommand{\timeder}[1]{\ensuremath{\partial_{t}{#1}}}
\newcommand{\volumeaverage}[1]{\ensuremath{\left\langle{#1}\right\rangle}_V}
\newcommand{\areaaverage}[1]{\ensuremath{\left\langle{#1}\right\rangle}_A}
\newcommand{\lineaverage}[1]{\ensuremath{\left\langle{#1}\right\rangle}_L}

% VARIABLES

\newcommand{\F}{\ensuremath{\tnsr F}}
\newcommand{\Fp}[1][]{\ensuremath{\tnsr F_\text{p#1}}}
\newcommand{\Fpo}[1][]{\ensuremath{\tnsr F_\text{p#10}}}
\newcommand{\Fpinv}[1][]{\ensuremath{\inverse{\tnsr F_\text{p#1}}}}
\newcommand{\Fpdot}[1][]{\ensuremath{\dot{\tnsr F}_\text{p#1}}}
\newcommand{\Fe}{\ensuremath{\tnsr F_\text{e}}}
\newcommand{\Lp}[1][]{\ensuremath{\tnsr L_\text{p#1}}}
\newcommand{\Lpo}[1][]{\ensuremath{\tnsr L_\text{p#10}}}
\newcommand{\E}{\ensuremath{\tnsr E}}
\newcommand{\GL}{\ensuremath{\tnsr E_\text{e}}}
\newcommand{\cauchy}[2]{\ensuremath{\sigma^{#1}_{#2}}}
\newcommand{\fPK}{\ensuremath{\tnsr P}}
\newcommand{\sPK}{\ensuremath{\tnsr S}}
\newcommand{\sPKcomponent}[1]{\ensuremath{S_{#1}}}

\newcommand{\dfedf}{\totalder{\Fe}{\F}}
\newcommand{\dsdf}{\totalder{\sPK}{\F}}
\newcommand{\dsdfe}{\totalder{\sPK}{\Fe}}
\newcommand{\dpdf}{\totalder{\fPK}{\F}}
\newcommand{\dlpdf}{\totalder{\Lp}{\F}}
\newcommand{\dlpds}{\totalder{\Lp}{\sPK}}
\newcommand{\explp}{\exp\left[-\alpha \Delta t\Lp\right]}
\newcommand{\explpo}{\exp\left[-\frac{\left(1-\alpha \right)\Delta t\Lpo}{2}\right]}
\newcommand{\dfpinvdf}{\totalder{\Fpinv}{\F}}
\newcommand{\dfpinvdt}{\totalder{\Fpinv}{\text{t}}}
\newcommand{\dfpinvdlp}{\totalder{\Fpinv}{\Lp}}



% MATH
\usepackage{systeme}


\begin{document}
\author{Madeleine Bignon, Pratheek Shanthraj, Joseph D. Robson}
\title{Full-field model for dynamic precipitation}
\maketitle

\noindent This is the documentation for the crystal plasticity constitutive law for DAMASK including dynamic precipitation available in:
 
\noindent  \url{https://github.com/LightForm-group/Damask-KWN}.  \\
 
\noindent The installation instructions for this version of DAMASK are provided elsewhere:
 
  \noindent  \url{https://lightform-group.github.io/wiki/software_and_simulation/kwn-damask}.
%\date{}                                           % Activate to display a given date or no date

\section{Kampmann-Wagner Model}

The KWN model used in this full field model is the one described in \cite{Bignon2022}. A ternary system is considered, with main element C and solute atoms A and B. The precipitate composition - considered as constant - is given by \hypertarget{stoichiometry} {A\textsubscript{x}B\textsubscript{y}C\textsubscript{z}}.

A number density distribution of precipitates, $\phi\left(r,t\right)$, is defined. 
The total number density of precipitates, $N$, is given by
%
\begin{equation}
\label{eq: totaldensity}
N\left(t\right) = \int_{0}^{\infty} \phi\left(r,t\right) \inc{r},
\end{equation}
%
the average radius, $R$, is given by
%
\begin{equation}
\label{eq: avgradius}
R\left(t\right) = \frac{1}{N\left(t\right)}\int_{0}^{\infty} r \phi\left(r,t\right) \inc{r},
\end{equation}
%
and the volume fraction, $f_{v}$, is given by
%
\begin{equation}
\label{eq: volfrac}
f_{v}\left(t\right) =\int_{0}^{\infty} \frac{4}{3} \pi r^3 \phi\left(r,t\right) \inc{r}.
\end{equation}
%
The conservation of precipitates is then represented by the following conservation law
%
\begin{equation}
\label{eq: conservation}
\partialder{\phi}{t}+ \partialder{\left[v \phi\right]}{r} = S
\end{equation}
%
where, $v\left(r,t\right)$ is the growth rate of the precipitates, and $S\left(r,t\right)$ is the source term representing the nucleation of precipitates.

\subsection{Initial distribution}
A pre-existing precipitate distribution is considered. The initial distribution $N(0)$ is assumed to follow a log-normal law, and is defined by its initial precipitate volume fraction \hypertarget{vf}{$f_{v}(0)$}, initial mean radius \hypertarget{r0}{$R(0)$} and standard deviation \hypertarget{sigma}{$\sigma$}\footnote{the log-normal distribution is sometimes defined by a dimensionless dispersion parameter $s$ defined as $s=\sigma/R(0)$} as \cite{Bignon2022}:
\begin{equation} \label{eq:number_density}
N(0)  =f_{v}(0)\frac{\int_{0}^{\infty} f(r) dr}{\int_{0}^{\infty} f(r) \frac{4}{3} \pi r^{3}dr}
\end{equation}
%
where $f$ is a function defined as:
%
\begin{equation} \label{eq:distribution_function}
f(r)=\frac{1}{\sqrt{2 \pi} \sigma} \exp\left[{-\frac{R(0)^{2}}{2 \sigma^{2}} \left(\ln(\frac{r}{R(0)})+ \frac{\sigma^{2}}{2R(0)^{2}}\right)^{2}}\right]
\end{equation}


\subsection{Precipitate growth}
The growth rate, $v\left(r,t\right)$ in \eref{eq: conservation} is assumed to be controlled by the slowest diffuser. Assuming the slowest diffuser is A,  $v\left(r,t\right)$ is given by:
\begin{equation} \label{eq:growth_rate}
v\left(r,t\right)=\frac{D}{r}\cdot \frac{\overline{x}_{ \tiny \textrm{A}} - x_{ \tiny \textrm{A}}^{r}}{x_{ \tiny \textrm{A}}^{\textrm pr} -  x_{ \tiny \textrm{A}}^{r} }
\end{equation}
where $D$ is the diffusion coefficient, calculated as detailed later,  $\overline{x}_{ \tiny \textrm{A}}$ is the average atomic fraction of A in the matrix, \hypertarget{xpr_a}{$x_{ \tiny \textrm{A}}^{\textrm pr}$} is the concentration in A in the precipitate ($x_{ \tiny \textrm{A}}^{\textrm pr} = \frac{x}{x+y+z}$), and $x_{ \tiny \textrm{A}}^{r}$ is the concentration in A at the interface between the matrix and a precipitate of radius $r$, calculated as the solution of the following system of equations \cite{ Nicolas2003,Bignon2022}: 
%

\begin{equation}\label{eq:interface}
 \begin{cases}
x\cdot(x_{ \tiny \textrm{B}}^{0}-x_{ \tiny \textrm{B}}^{r}) =y\cdot( x_{ \tiny \textrm{A}}^{0}  -x_{ \tiny \textrm{A}}^{r}) \\
\exp\left[(\frac{2 \gamma  v_{\textrm m}^{pr}}{r  RT} +\frac{\Delta G_E\cdot v_{\textrm m}^{pr}}{RT} )\cdot(x+y+z) \right] \cdot (x_{ \tiny \textrm{A}}^{\infty})^{x} \cdot (x_{ \tiny \textrm{B}}^{\infty})^{y}  =  (x_{ \tiny \textrm{A}}^{r})^{x} \cdot (x_{ \tiny \textrm{B}}^{r})^{y}    \\
\end{cases} 
\end{equation}
where \hypertarget{strain_energy}{$\Delta G_E$} is the elastic strain energy,  \hypertarget{xb}{$x_{ \tiny \textrm{A}}^{0}$} and \hypertarget{xc}{$x_{ \tiny \textrm{B}}^{0}$} are the bulk concentrations of the alloy in species A and B, respectively;  \hypertarget{xaeq}{$x_{ \tiny \textrm{A}}^{\infty}$} and \hypertarget{xbeq}{$x_{ \tiny \textrm{B}}^{\infty}$} are the equilibrium concentrations in A and B, respectively; $x_{ \tiny \textrm{B}}^{r}$ - which is the second variable of the system in \eref{eq:interface} -  is the concentration at the interface between the matrix and a precipitate of radius $r$, and \hypertarget{vmol}{$v_{\textrm m}^{pr}$} is the molar volume of the precipitate. 


\subsection{Source term}
Using classical nucleation theory, the source term in \eref{eq: conservation} can be expressed as
%
\begin{equation}
\label{eq: source}
S = \delta\left(r - r_c\right) I
\end{equation}
%
where $r_c\left(t\right)$ is the critical nucleus size, and $I\left(t\right)$ is the nucleation rate.
The critical nucleus size $r_c(t)$ is calculated as the radius such that:
 \begin{equation}
 v(r_c, t) = 0
 \end{equation}
 
 The nucleation rate is given by
%
\begin{equation}
\label{eq: nucleation}
I\left(t\right) = N_s Z \beta \exp{\left[-\frac{4\pi\gamma r_c^2}{3\kB T}\right]} 
\end{equation}
%
where $N_s$ is the number density of nucleation sites, calculated as 
\begin{equation}
N_s=\frac{\overline{x}_{ \tiny \textrm{A}}+\overline{x}_{ \tiny \textrm{B}}}{V}
\end{equation}
where \hypertarget{vat}{$V$} is the atomic volume of the matrix.
In \eref{eq: nucleation}, the Zeldovich factor, $Z$, is given by
%
\begin{equation}
\label{eq: soluteconservation}
Z = \frac{V}{2\pi r_c^2} \sqrt{\frac{\gamma}{\kB T}},
\end{equation}
%
$\beta$ is given by
%
\begin{equation}
\label{eq: beta}
\beta = 4\pi D (\overline{x}_{ \tiny \textrm{A}}+\overline{x}_{ \tiny \textrm{B}}) \frac{r_c^2}{a^4},  
\end{equation}
where \hypertarget{latticeparameter}{$a$} is the lattice parameter of the product phase.
 
 
 \subsection{Solute conservation}
The precipitate solute concentration in element $i$ (where i=A or B), $x_{ \tiny \textrm{i}}^{\textrm pr}$ , is assumed to be constant, and the matrix solute concentration satisfies:
%
\begin{equation}
\label{eq: soluteconservation}
{x}_{ \tiny {i}}^{0} = f_v(t) x_{ \tiny \textrm{i}}^{\textrm pr} + \left(1 - f_v(t)\right)\overline{x}_{ \tiny \textrm{i}}(t),
\end{equation}
 

%

%
\section{Crystal plasticity model}

A crystal plasticity constitutive model is used and the plastic strain rate is given by \cite{Roters2019a}:
%
\begin{equation}
\label{eq: lp}
\Lp = \sum_{\alpha = 1}^{12} \dot{\gamma}^{\alpha} n^{\alpha} \otimes s^{\alpha} 
\end{equation}
where the indices $\alpha$ refer to the 12 FCC slip systems, defined by the slip plane normal $n^{\alpha}$ and the slip direction $s^{\alpha}$, and $\dot{\gamma}^{\alpha}$ is the slip rate.
The slip rates are given by an empirical power law  \cite{Roters2010}:
%
\begin{equation}
\label{eq:sliprate}
\dot{\gamma}^{\alpha} = \dot{\gamma}_0\left(\frac{\tau^{\alpha}}{\tau_{*}^{\alpha} }\right)^{n}
\end{equation}
%
where \hypertarget{gamma_0}{$\dot{\gamma}_0$} and \hypertarget{nsl}{$n$} are constants, $\tau^{\alpha}$ is the resolved shear stress on the slip system, and the slip resistance, $\tau_{*}^{\alpha}$, is given by \cite{Deschamps1998c}:
%}
\begin{equation}
\label{eq:crss}
\tau_{*}^{\alpha} = \tau_{s} + \sqrt{(\tau_{p})^{2}+(\tau_{d}^{\alpha})^{2}}
\end{equation}
where $\tau_{s}$, $\tau_{p}$ and $\tau_{d}^{\alpha}$ are the contributions of the atoms in solid solution, of the reinforcing precipitates, and of the dislocations, respectively. $\tau_{s}$ is given by:
\begin{equation}
\label{eq:tau_s}
\tau_{s} = k_{s} \cdot (\overline{x}_{ \tiny \textrm{A}}+\overline{x}_{ \tiny \textrm{B}})^{2/3}
\end{equation}
where \hypertarget{ks}{$k_{s}$} is an empirical constant \cite{Deschamps1998c}. The precipitation hardening contribution $\tau_{p}$, is given by \cite{Deschamps1998c}:
\begin{equation}
\label{eq:precipitatehardening}
\tau_{p}= \mu \sqrt{\frac{3 f_{v}}{2 \pi}} \cdot  \frac{k_{p}}{ R  \cdot \sqrt{r_{t}} }\left( \int_{0}^{r_{t}}  r \cdot \phi\left(r,t\right) \inc{r} + \int_{r_{t}}^{\infty} r_{t}  \cdot \phi\left(r,t\right) \inc{r} \right)^{3/2}
\end{equation}
%
where \hypertarget{mu}{$\mu$} is the shear modulus, \hypertarget{kp}{$k_{p}$} is an empirical constant, and \hypertarget{rt}{$r_{t}$} is the transition radius between shearing and Orowan looping \cite{Hull2011}. 
In \eref{eq:crss}, $\tau_{d}^{\alpha}$ is the resistance to slip due to forest dislocations, and the hardening rate is given by \cite{Roters2010}:
\begin{equation}
\label{eq:strainhardening}
\dot{\tau}_d^{\alpha} = h_0 \sum_{\beta = 1}^{N}q_{\alpha\beta}\ \dot{\gamma}^{\beta}\left[1 - \frac{\tau_d^{\beta}}{\tau_{\infty}}\right]^{m},
\end{equation}
where \hypertarget{h0}{$h_{0}$}, \hypertarget{tauinfinity}{$\tau_{\infty}$} and \hypertarget{m}{$m$} are hardening parameters, and \hypertarget{q}{$q_{\alpha\beta}$} is the latent hardening between the slip systems $\alpha$ and $\beta$. The initial value of ${\tau}_d^{\alpha}$ (considered as identical for all slip systems and labelled \hypertarget{tau_d_alpha}{${\tau}_d(0)$}) is a model input.




\section{Excess vacancy model}
Plastic deformation produces vacancies which enhance diffusivity and accelerate precipitation kinetics. A phenomenological model \cite{Robson2020, Militzer1994, Bignon2022} is used to calculate the excess vacancy concentration as a function of temperature and deformation conditions. The rate of generation and annihilation of vacancies is given by
%
\begin{equation}
\label{eq: vacancyrate}
\dot{c}_\text{V} = \chi \frac{\hyperlink{vat}{V}}{E_\text{V}} \sum_{\alpha = 1}^{N} \dot{\gamma}^{\alpha} \tau^{\alpha}  + 0.5\cdot\frac{c_{j}V}{4b^{3}}  \sum_{\alpha = 1}^{N} \dot{\gamma}^{\alpha}   - \frac{D_\text{V}}{L^2} c_\text{V},
\end{equation} 
where \hypertarget{generation}{$\chi$} is the fraction of plastic work converted into vacancies;  \hypertarget{EV}{$E_\text{V}$}, is the vacancy formation energy; $D_\text{V} = \hypertarget{DV0}{D_\text{V0}} \exp\left[-\frac{\hypertarget{MV}{M_\text{V}}}{\kB T}\right]$ is the vacancy diffusion coefficient in terms of the vacancy migration energy, $M_\text{V}$; \hypertarget{b}{$b$} the burgers vector of screw dislocations; and $c_{j}$ is the dislocation jog concentration, calculated as $c_j=\exp\left[\frac{-E_j}{\kB T}\right]$, where \hypertarget{Ej}{$E_j$} is the jog formation energy.
$L$ is the average spacing between vacancy sinks, and is given by the minimum of dislocation spacing and grain size \hyperlink{dg}{$d_{g}$}
%
\begin{equation}
\label{eq: vacancysink}
L = \min\left[\frac{0.27 \mu b}{\max(\tau_d^{\alpha}) \cdot \kappa^{2}}, d_g\right].
\end{equation} 
%
where \hypertarget{kappa}{$\kappa$} is a constant between 1 and 10 related to the dislocation arrangement (\cite{Bignon2022, Robson2020}).
The solute diffusion coefficient $D$ in \eref{eq: beta} and \eref{eq:growth_rate} is related to the vacancy concentration through
%
\begin{equation}
\label{eq: diffusion}
D = D_{\text{th}}(1+\frac{c_v-c{_\text{eq}}}{c_{\text{eq}}})
\end{equation} 

where $c{_\text{eq}}$ is the equilibrium vacancy concentration, calculated as $c{_\text{eq}}=23\exp[-\frac{E_f}{\kB T}]$ \cite{Bignon2022}, and $D_{\text{th}}$ is the solute diffusion coefficient in the absence of any deformation, calculated as $D_{\text{th}}=\hyperlink{D0}{D_0}\exp(-E_s/\kB T)$, where \hyperlink{Es}{$E_s$} is the solute migration energy. It is assumed that $c_v(0)=c{_\text{eq}}$.

%



\section{To improve the model}
In the future, the influence of dislocation density on the nucleation site density, $N_s$ in \eref{eq: nucleation}, through the elastic misfit energy, $\Delta G_E$, and the solute diffusion coefficient, $D$, through pipe diffusion, might be worth looking into.

\subsection{Mass Transport}

The solute and vacancy balance in \eref{eq: soluteconservation} and \eref{eq: vacancyrate} respectively local.
In addition long-range transport can be incorporated by including diffusional fluxes
%
\begin{align}
\label{eq: diffusion}
\dot{c}_\text{V} = \nabla D_\text{V} \nabla c_\text{V} + \chi \frac{\Omega}{E_\text{V}} \sum_{\alpha = 1}^{N} \dot{\gamma}^{\alpha} \tau^{\alpha} - \frac{D_\text{V}}{L^2} c_\text{V} \\
\dot{c}_\text{M} = \nabla D_\text{M} \nabla c_\text{M} + \frac{\dot{f}}{1-f}\left(c_\text{M} - c_\text{P}\right)
\end{align} 

\subsection{Decreasing calculation time}
Currently, the equilibrium composition at the interface between matrix and precipitate is calculated at each time step for all bins of the distribution. As the equilibrium concentration at the interface does not depend on time, the code could be modified to calculate the equilibrium composition only at the beginning.
\subsection{Improve readability}
Readability might be improved by converting all units in the \texttt{phase\char`_mechanical\char`_plastic\char`_kwnpowerlaw.f90} source file to international unit system (for now, several variables are normed).
\section{Model inputs}

\begin{center}
\begin{tabular}{llr}
\textbf{Symbol} & \textbf{Variable} & \textbf{Unit} \\
\hline
\hyperlink{m}{m} & \texttt{a\char`_sl} & -\\
\hline
\hyperlink{vat}{$V$} & \texttt{atomic\char`_volume} & \si{\meter^{3}} \\
\hline
\hyperlink{b}{$b$} &  \texttt{burgers\char`_vector}  &\si{\meter} \\
\hline
[\hyperlink{xb}{$x_{ \tiny \textrm{A}}^{0}$}, \hyperlink{xc}{$x_{ \tiny \textrm{B}}^{0}$}] & \texttt{c0\char`_matrix} & at \\
\hline
[ \hyperlink{xaeq}{$x_{ \tiny \textrm{A}}^{\infty}$}, \hyperlink{xbeq}{$x_{ \tiny \textrm{B}}^{\infty}$}] & \texttt{ceq\char`_matrix} & at \\
\hline
[\hyperlink{xpr_a}{$x_{ \tiny \textrm{A}}^{\textrm pr}$},  \hyperlink{xpr_b}{$x_{ \tiny \textrm{B}}^{\textrm pr}$}]& \texttt{ceq\char`_precipitate} & at \\
\hline
\hyperlink{kappa}{$\kappa$} &\texttt{dislocation\char`_arrangement} &-\\
\hline
\hyperlink{gamma_0}{$\dot{\gamma}_0$}  &  \texttt{dot\char`_gamma\char`_0\char`_sl} & \si{\meter^{-1}}\\
\hline
\hyperlink{interfacial_energy}{$\gamma$}& \texttt{gamma\char`_coherent} & \si{\joule \meter^{-2}} \\
\hline
\hyperlink{h0}{$h_{0}$} &\texttt{h\char`_0\char`_sl\char`_sl}  & \si{\pascal} \\
\hline
 \hyperlink{q}{$q_{\alpha\beta}$}  &\texttt{h\char`_sl\char`_sl} & - \\
\hline
 \hyperlink{r0}{$R(0)$} &\texttt{initial\char`_mean\char`_radius}  & \si{\meter} \\
 \hline
 \hyperlink{vf}{$f_{v}(0)$} & \texttt{initial\char`_volume\char`_fraction} &-\\
 \hline

 \hyperlink{Ej}{$E_j$} & \texttt{jog\char`_formation\char`_energy} & eV \\
\hline
\hyperlink{latticeparameter}{$a$} & \texttt{lattice\char`_parameter} & \si{\meter} \\
\hline
\hyperlink{strain_energy}{$\Delta G_E$} & \texttt{misfit\char`_energy} & \si{\joule \meter^{-3}} \\
\hline
\hyperlink{vmol}{$v_{\textrm m}^{pr}$} &  \texttt{molar\char`_volume} & \si{\meter^{3}\mole^{-1}}\\
\hline
\hyperlink{nsl}{$n$} &  \texttt{n\char`_sl} & - \\
\hline
\hyperlink{kp}{$k_p$} & \texttt{precipitate\char`_strength\char`_constant} & - \\
\hline
\hyperlink{mu}{$\mu$} &  \texttt{shear\char`_modulus} & \si{\pascal} \\
\hline
\hyperlink{D0}{$D_0$} &\texttt{solute\char`_diffusion0} &\si{\meter^{2} \second^{-1}} \\
\hline
\hyperlink{Es}{$E_s$} & \texttt{solute\char`_migration\char`_energy} & eV \\
\hline
 \hyperlink{ks}{$k_s$} & \texttt{solute\char`_strength} & - \\
\hline
\hyperlink{sigma}{$\sigma$} & \texttt{standard\char`_deviation} & \si{\meter} \\ 
 \hline
\hyperlink{stoichiometry}{[x,y,z]} & \texttt{stoechiometry} & - \\ 
\hline
\hyperlink{rt}{$r_t$} & \texttt{transition\char`_radius} & \si{\meter} \\
\hline
\hyperlink{tau_d_alpha}{${\tau}_d(0)$} &\texttt{xi\char`_0\char`_sl}  & \si{\pascal}\\
\hline
\hyperlink{tauinfinity}{$\tau_{\infty}$} &\texttt{xi\char`_inf\char`_sl}  & \si{\pascal}\\
\hline
\hyperlink{DV0}{$D_\text{V0}$} &\texttt{vacancy\char`_diffusion0} &\si{\meter^{2} \second^{-1}} \\
\hline
\hyperlink{EV}{$E_\text{V}$} & \texttt{vacancy\char`_energy}& eV \\
\hline 
\hyperlink{generation}{$\chi$} & \texttt{vacancy\char`_generation} & - \\
\hline


\hyperlink{MV}{$M_\text{V}$} &\texttt{vacancy\char`_migration\char`_energy} &eV\\
\hline


\hyperlink{dg}{$d_{g}$} & \texttt{vacancy\char`_sink\char`_spacing}& \si{\meter}\\
\hline


\end{tabular}
\end{center}


\bibliographystyle{unsrtnat}
\bibliography{fullfield-kwn}
\end{document}  